\section{Möbiusavbildningar}

\subsection{Spegelpunkter}
Detta är väldigt centralt, ser man frasen "med nödvändighet" i ett lösningsförslag så är det detta som de menar. Det fungerar såhär, om man har en punkt
\begin{align*}
	w(z) = a \iff w(z*) = a^*
\end{align*}
Där dessa spegelpunkter är med avseende på området man har. $z$ och $z^*$ är spegelpunkter till varandra med avseende på vad z-planet har för område, till exempel en cirkel $|z-c| = r$ eller en linje.  Dessa kan räknas ut genom
\begin{align*}
	|z-c| \cdot |z^*-c| = r^2 = |z-c|^2	
\end{align*}
ur detta  samband kan man enkelt räkna ut z^*. \\

Linjer är ännu enklare, då bara speglar man med avseende på mittpunktsnormalen.
